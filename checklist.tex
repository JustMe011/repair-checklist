\documentclass[a4paper,12pt,twoside]{article}
\usepackage[italian]{babel}
\usepackage[utf8]{inputenc}
\usepackage[T1]{fontenc}

\usepackage{bookman}
%\usepackage{libertine}
\renewcommand{\familydefault}{\sfdefault}
\usepackage[paper=a4paper,top=0.8cm,bottom=1.2cm,right=2cm,left=2cm]{geometry} % margini
\usepackage{graphicx}
\usepackage{titlesec}
\usepackage{framed,wrapfig}

\newcommand{\rulespace}[1]{\ \ \textit{\small #1} \hrulefill\par}
\renewcommand{\r}{\rulespace{Risultato:}}
\renewcommand{\c}{\framebox(9,9){}\, }
\newcommand{\ci}{\hspace{1cm}\c} % \c indentato
\usepackage{parskip}

\begin{document}
\thispagestyle{empty}
{\Large\textsc{Riparazione}}\par
\begin{wrapfigure}{l}{4 cm}
	\vspace{-0.5cm}
	\begin{framed}
		\begin{centering}
		{\footnotesize Codice inventario\par}
		\vspace{2.4cm}
		\end{centering}		
	\end{framed}
	\vspace{-1.4cm}
\end{wrapfigure}
Sintomi: \hrulefill\par
\hrulefill\par
\c Controllare cavi inseriti bene\r
\c Controllare cavo VGA in porta giusta\r
\c Avvio con interno scollegato\r
\c Avvio con periferiche scollegate (anche tastiera)\r
\c Avvio da CD/DVD\r
\c Avvio da floppy\r
\c Avvio con scheda di debug\r
\c Avvio senza scheda di debug\r
\c CMOS reset (togliendo batteria o col jumper) \r
\c Cambiare impostazioni nel BIOS\r
\c Test alimentatore\r
\c Sostituzione alimentatore\r
\c Sostituzione scheda grafica\r
\c Aggiunta/rimozione (se non c'era/c'era) scheda grafica\r
\c Sostituzione condensatori esplosi\r
\c Sostituzione altri componenti\r
\rulespace{Operatore assegnato:}
{\Large\textsc{Ordinaria amministrazione}}\par
\c Inserito almeno 1 GB di RAM\par
\c Inserito HDD da almeno 40 GB (o più di uno se sono da meno)\par
\c Batteria CMOS carica\par
\c Cambiata pasta termica\par
\c Xubuntu LTS installato\par
\ci Partizioni root, home e swap separate\par
\ci Username \textit{weee}, password \textit{open}\par
\ci Hostname \textit{pc-nomedelpc}, e.g. il D2 ha hostname \textit{pc-D2}\par
\ci Spuntato ``scarica aggiornamenti'' e ``installa software di terze parti''\par
\ci \textit{sudo apt update}, poi \textit{sudo apt upgrade}\par
\ci VLC installato\par
\ci Lingua italiana impostata da tool di sistema\par
\ci Impostare fuso orario \textit{Europe/Rome} da orologio\par
\ci \textit{timedatectl set-ntp true}\par
\c Tastiera, mouse e schermo appaiati per spedizione a MRF
\end{document}
