\documentclass[a4paper,12pt,twoside]{article}
\usepackage[italian]{babel}
\usepackage[utf8]{inputenc}
\usepackage[T1]{fontenc}

\usepackage{bookman}
%\usepackage{libertine}
\renewcommand{\familydefault}{\sfdefault}
\usepackage[paper=a4paper,top=0.8cm,bottom=1.2cm,right=2cm,left=2cm]{geometry} % margini
\usepackage{graphicx}
\usepackage{titlesec}
\usepackage{framed,wrapfig}

\newcommand{\rulespace}[1]{\ \ \textit{\small #1} \hrulefill\par}
\renewcommand{\r}{\rulespace{Risultato:}}
\renewcommand{\c}{\framebox(9,9){}\, }
\usepackage{parskip}

\begin{document}
\thispagestyle{empty}
{\Large\textsc{Riparazione}}\par
\begin{wrapfigure}{l}{4 cm}
	\vspace{-0.5cm}
	\begin{framed}
		\begin{centering}
		{\footnotesize Codice inventario\par}
		\vspace{2.4cm}
		\end{centering}		
	\end{framed}
	\vspace{-1.4cm}
\end{wrapfigure}
Sintomi: \hrulefill\par
\hrulefill\par
\c Controllare cavi inseriti bene\r
\c Controllare cavo VGA in porta giusta\r
\c Avvio con interno scollegato\r
\c Avvio con periferiche scollegate (anche tastiera)\r
\c Avvio da CD/DVD\r
\c Avvio da floppy\r
\c Avvio con scheda di debugging\r
\c Avvio senza scheda di debugging\r
\c CMOS reset (togliendo batteria o col jumper) \r
\c Cambiare impostazioni nel BIOS\r
\c Test alimentatore\r
\c Sostituzione alimentatore\r
\c Sostituzione scheda grafica\r
\c Aggiunta/rimozione (se non c'era/c'era) scheda grafica\r
\c Sostituzione condensatori\r
\c Sostituzione altri componenti\r
\rulespace{Operatore assegnato:}

\end{document}
